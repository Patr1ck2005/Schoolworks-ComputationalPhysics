%! Author = Kiddo_Tsukuyomi
%! Date = 

% Preamble
\documentclass[UTF8]{ctexart}

% Packages
\usepackage{geometry}
\usepackage{listings}
\usepackage{xcolor}
\usepackage{graphicx}
\usepackage{amsmath}
\usepackage{subcaption}

% Settings
\lstset{
    backgroundcolor=\color{gray!5}, % Code block background color
    basicstyle=\ttfamily\small, % Font style: monospaced, small size
    columns=fullflexible, % Allow line wrapping, no restrictions on column number
    keepspaces=true, % Preserve spaces
    language=[LaTeX]TeX, % Code language is LaTeX
    breaklines=true, % Allow automatic line breaking
    breakatwhitespace=true, % Break at whitespace only
    postbreak=\mbox{\textcolor{red}{$\hookrightarrow$}\space}, % Mark break position with red arrow
    commentstyle=\color{gray}\itshape, % Comment style: gray italic
    stringstyle=\color{orange}, % String style: orange
    keywordstyle=\color{blue}, % Keyword style: blue
    showstringspaces=false, % Do not show spaces within strings
    numbers=left, % Display line numbers on the left
    numberstyle=\scriptsize, % Line number font size: footnote size
    numbersep=8pt, % Distance between line numbers and code: 8pt
    frame=single, % Add single line border
    frameround=tttt, % Rounded corners for the frame
    rulecolor=\color{black}, % Frame color: black
    extendedchars=true, % Allow the use of extended character sets
    inputencoding=utf8, % Input encoding: UTF-8
    literate={~} {$\sim$}{1}, % Handle tilde symbol display
    escapeinside=``, % Use backticks as an escape region, to avoid garbled characters when using Chinese text
    xleftmargin=2em, % Left margin of the code block
    xrightmargin=2em, % Right margin of the code block
    aboveskip=1em, % Space above the code block
    framexleftmargin=2em, % Left margin of the frame
}

% Document
\begin{document}
    \title{\textbf{HW8}}
    \author{白博臣,何骐多,夏营}
    \date{}
    \maketitle


    \section{Problem1}\label{sec:1}
    \subsection{二阶}\label{subsec:1.1}


    令
    \begin{equation*}
        x_n=x_0+nh
    \end{equation*}

    二阶向前差分公式:
    \begin{equation*}
        f'' (x_0 ) = \frac{(f(x_0 ) - 2f(x_1 ) + f(x_2))}{h^2} +O(h)
    \end{equation*}

    二阶中心差分公式:
    \begin{equation*}
        f'' (x_0 ) = \frac{(f(x_(-1) ) - 2f(x_0 ) + f(x_1))}{h^2} +O(h^2)
    \end{equation*}

    二阶向后差分公式:
    \begin{equation*}
        f'' (x_0 ) = \frac{(f(x_(-2) ) - 2f(x_(-1) ) + f(x_0 ))}{h^2 }+O(h)
    \end{equation*}

    \subsection{三阶}\label{subsec:1.2}


    三阶向前差分公式:
    \begin{equation*}
        f''' (x_0 ) = \frac{(-f(x_0 ) + 3f(x_1 ) - 3f(x_2 ) + f(x_3 ))}{h^3} +O(h)
    \end{equation*}

    三阶中心差分公式:
    \begin{equation*}
        f''' (x_0 ) = \frac{(-f(x_(-2) ) + 2f(x_(-1) ) - 2f(x_1 ) + f(x_2 ))}{2h^3} +O(h^2 )
    \end{equation*}

    三阶向后差分公式:
    \begin{equation*}
        f''' (x_0 ) = \frac{(-f(x_(-3) ) + 3f(x_(-2) ) -3 f(x_(-1) ) + f(x_0 ))}{h^3} +O(h)
    \end{equation*}

    \subsection{四阶}\label{subsec:1.3}


    四阶向前差分公式:
    \begin{equation*}
        f^{iv} (x_0 ) = \frac{(f(x_0 ) - 4f(x_1 ) + 6f(x_2 ) - 4f(x_3 ) + f(x_4))}{h^4} +O(h)
    \end{equation*}

    四阶中心差分公式:
    \begin{equation*}
        f^{iv} (x_0 ) = \frac{(f(x_(-2) ) - 4f(x_(-1) ) + 6f(x_0) - 4f(x_1 ) + f(x_2 ))}{h^4} +O(h^2 )
    \end{equation*}

    四阶向后差分公式:
    \begin{equation*}
        f^{iv} (x_0 ) = \frac{(f(x_(-4)) - 4f(x_(-3) ) + 6f(x_(-2) ) - 4f(x_(-1) ) + f(x_0 ))}{h^4} +O(h)
    \end{equation*}


    \subsection{五阶}\label{subsec:1.4}

    三阶向前差分公式:
    \begin{equation*}
        f^v (x_0 ) = \frac{(f(x_5 ) - 5f(x_4 ) + 10f(x_3 ) - 10f(x_2 ) + 5f(x_1 ) - f(x_0))}{h^5} +O(h)
    \end{equation*}

    三阶中心差分公式:
    \begin{equation*}
        f^v (x_0 ) = \frac{(-f(x_(-3) ) + 4f(x_(-2) ) - 5f(x_(-1) ) + 5f(x_1 ) - 4f(x_2 ) + f(x_3))}{2h^5}+O(h^2 )
    \end{equation*}

    三阶向后差分公式:
    \begin{equation*}
        f^v (x_0 ) = \frac{(f(x_(-5) ) - 5f(x_(-4) ) + 10f(x_(-3) ) - 10f(x_(-2) ) + 5f(x_(-1) ) - f(x_0))}{h^5} +O(h)
    \end{equation*}

    \newpage


    \section{Problem2}\label{sec:2}
    利用 \(o(h^2)\) 的中心差分公式计算一阶导数,使用 Java 代码实现如下
    \begin{lstlisting}[language=Java, breaklines = true,label={lst:lstlisting2}]
import static java.lang.StrictMath.*;

interface function {
    double cal(double x);
}

public class Derivative {
    public static void main(String[] args) {
        int n = 4;
        double[] h = new double[n];
        for (int i = 0; i < n; i++) {
            h[i] = pow(10, -1 * (i + 1));
        }
        function f = (x) -> x * exp(x);
        function d1f = (x) -> (x + 1) * exp(x);
        double f01, f02, f11, f12, d1f1;
        double d1f0 = d1f.cal(2);

        for (double v:h) {
            f01 = f.cal(2 + v);
            f02 = f.cal(2 + 2 * v);
            f11 = f.cal(2 - v);
            f12 = f.cal(2 - 2 * v);
            d1f1 = (-f02 + 8 * f01 - 8 * f11 + f12) / 12 / v;

            double err = abs((d1f1 - d1f0) / d1f0);

            System.out.printf("h = %f\n",v);
            System.out.printf("f'0(2.0)   = %.16f, f'(2.0)   = %.16f, relative error = %.16f\n",d1f0, d1f1, err);
        }
    }
}
    \end{lstlisting}

    运行结果如下
    \begin{figure}[h]
        \centering
        \includegraphics[height=0.15 \textheight]{Derivative1}
        \caption{Derivative 运行结果}\label{fig:figure2}
    \end{figure}

    据计算,在 \(h = 0.001\) 时,相对误差较小,是该范围内的最佳。

    当 \(h\) 缩小时,相对误差会先缩小在增大,当 \(h < 10^{-4}\) 时,相对误差会随 \(h\) 缩小而增大。

    \newpage


    \section{Problem3}\label{sec:3}
    利用 \(o(h^2)\) 的中心差分公式计算二阶导数,使用 Java 代码实现如下
    \begin{lstlisting}[language=Java, breaklines = true,label={lst:lstlisting3}]
import static java.lang.StrictMath.*;

interface function {
    double cal(double x);
}

public class Derivative {
    public static void main(String[] args) {
        int n = 4;
        double[] h = new double[n];
        for (int i = 0; i < n; i++) {
            h[i] = pow(10, -1 * (i + 1));
        }
        function f = (x) -> x * exp(x * x);
        function d2f = (x) -> (4 * x * x * x + 6 * x) * exp(x * x);
        double f00, f01, f02, f11, f12, d2f1;
        double d2f0 = d2f.cal(2);

        for (double v : h) {
            f00 = f.cal(2);
            f01 = f.cal(2 + v);
            f02 = f.cal(2 + 2 * v);
            f11 = f.cal(2 - v);
            f12 = f.cal(2 - 2 * v);
            d2f1 = (-f02 + 16 * f01 - 30 * f00 + 16 * f11 - f12) / 12 / v / v;

            double err = abs((d2f1 - d2f0) / d2f0);

            System.out.printf("h = %f\n",v);
            System.out.printf("f'0(2.0)   = %.16f, f'(2.0)   = %.16f, relative error = %.16f\n",d2f0, d2f1, err);
        }
    }
}
    \end{lstlisting}

    运行结果如下
    \begin{figure}[h]
        \centering
        \includegraphics[height=0.15 \textheight]{Derivative2}
        \caption{Derivative 运行结果}\label{fig:figure3}
    \end{figure}

    据计算,在 \(h = 0.001\) 时,相对误差较小,是该范围内的最佳。

    当 \(h\) 缩小时,相对误差会先缩小在增大,当 \(h < 10^{-4}\) 时,相对误差会随 \(h\) 缩小而增大。

    \newpage


    \section{Problem4}\label{sec:4}

    \subsection{a}\label{subsec:4.1}
    使用 Python 代码实现如下
    \begin{lstlisting}[language=Python, breaklines = true,label={lst:lstlisting4.1}]
# `设置函数f(x)=x(x-1)`
def f(x):
    return x * (x - 1)


# `设置求解导数的公式`
def derivative_f(x, delta):
    return (f(x + delta) - f(x)) / delta


# `设置导数函数 f'(x)=2x-1`
def real_df(x):
    return 2 * x - 1


# `计算并且输出结果`
approximation_df = derivative_f(1, 1e-2)
exact_df = real_df(1)

print('`近似值:`', '\n', approximation_df)
print('`准确值:`', '\n', exact_df)
    \end{lstlisting}

    运行结果如下
    \begin{figure}[h]
        \centering
        \includegraphics[height=0.12 \textheight]{Prob4a}
        \caption{HW8\_4\_a 运行结果}\label{fig:figure4.1}
    \end{figure}

    结论:

    近似值与准确值并不完全相同, 这是因为此时的 \(\delta\) 的取值 \(\delta = 10^{-2}\) 并没有达到 \(\lim{\delta \rightarrow 0}\) ,所以此时是用的函数上 \(x = 0.5\) 与 \(x = 0.5 + \delta\) 两点的割线斜率来代替的导数值,与实际结果会存在误差。

    \subsection{b}\label{subsec:4.2}
    加入一个 for 循环语句来实现对多个 \(\delta\) 值的运算,使用 Python 代码实现如下
    \begin{lstlisting}[language=Python, breaklines = true,label={lst:lstlisting4.2}]
import numpy as np


# `设置函数f(x)=x(x-1)`
def f(x):
    return x * (x - 1)


# `设置求解导数的公式`
def derivative_f(x, delta):
    return (f(x + delta) - f(x)) / delta


# `设置导数函数 f'(x)=2x-1`
def real_df(x):
    return 2 * x - 1


# `设置不同的delta值来计算导数并输出结果`
log_delta = np.linspace(-2, -18, 9)
delta = 10 ** log_delta
exact_df = real_df(1)
print('`准确值为`', '\n', exact_df)
for i in range(len(delta)):
    approximation_df = derivative_f(1, delta[i])
    print('`当`delta={}`时`'.format(delta[i]), '\n', approximation_df)
    \end{lstlisting}

    运行结果如下
    \begin{figure}[h]
        \centering
        \includegraphics[height=0.64 \textheight]{Prob4b}
        \caption{HW8\_4\_b 运行结果}\label{fig:figure4.2}
    \end{figure}

    \newpage
    结论:

    当 \(\delta\) 值在 \(10^{-2}, 10^{-8} \) 中时,随着 \(\delta\) 的减小,近似值逐渐逼近真实值,而在 \(\delta > 10^{-8}\) 后,随着 \(\delta\) 的减小,近似值与真实值的误差逐渐增大,当 \(\delta > 10^{-16}\) 时,近似值为 \(0\) 。

    现对上述异常现象做出解释。

    利用计算机并通过公式 \(\frac{df(x)}{dx} = \lim_{\delta \rightarrow 0}{\frac{f(x + \delta)-f(x)}{\delta}}\) 估计近似值时,其同时会产生两种误差,一类是截断误差(truncation error),其与 \(\delta\) 的值成正相关;另一类是由于算法问题以及机器精度导致的舍入误差(round-off error),其与机器精度成正相关,与 \(\delta\) 成负相关。这两类误差的和组成了这一计算公式导致的总误差,并且由于两类误差与 \(\delta\) 的相关性不同,这就导致当 \(\delta\) 减小时,截断误差在减小,而舍入误差在增大。结合 lecture 8 可以写出两类误差具体的表达式。

    截断误差:
    \begin{equation*}
        e\_tru = \delta / 2 \abs{f''(x)}
    \end{equation*}

    舍入误差:
    \begin{equation*}
        e\_rou = (2c \abs{f(x)}) / \delta
    \end{equation*}
    其中 \(c\) 为机器精度

    于是总误差为:
    \begin{equation*}
        e = e\_tru + e\_rou = \delta / 2 \abs{f''(x)} + (2c \abs{f(x)}) / \delta
    \end{equation*}

    当 \(\delta\) 值在 \(10^{-2},10^{-8}\) 中时,截断误差的影响占主导地位,所以此时总误差会随着 \(\delta\) 的减小而减小,而当 \(\delta > 10^{-8}\) 时,截断误差趋近于 \(0\) ,舍入误差的影响占主导地位了,所以此时总误差会随着 \(\delta\) 的减小而增大。而当 \(\delta > 10^{-16}\) 时,此时, \(\delta\) 接近于甚至小于机器精度 \(c\) 值,于是 \(f(x + \delta) - f(x)\) 在计算机的运算过程被直接舍入为 \(0\) 了,所以此时近似值为 \(0\) 。

    \subsection{c}\label{subsec:4.3}
    首先生成一个在区间 \([0, 20]\) 有 \(200\) 项的等差数列 \(log_{10}{h}\) ,而后复用上一问代码,计算出不同 \(h\) 值下的近似值,而后与真实值做差得到误差。再对误差做一次 \(log_{10}\) 的运算,再利用题意所给的
    \begin{equation*}
        g(h) = \frac{h}{2} \sin{0.5} + \epsilon'' \frac{2}{h} \sin{0.5}
    \end{equation*}
    画出所需的 \(log10-log10\) 图像。

    使用 Python 代码实现如下
    \begin{lstlisting}[language=Python, breaklines = true,label={lst:lstlisting4.3}]
import matplotlib.pyplot as plt
import numpy as np


# `设置函数f(x)=sin(x)`
def f(x):
    return np.sin(x)


# `设置求解导数的公式`
def derivative_f(x, delta):
    return (f(x + delta) - f(x)) / delta


# `设置导数函数 f'(x)=cos(x)`
def real_df(x):
    return np.cos(x)


# `设置g(h)=h*sin(0.5)/2+2c*sin(0.5)/h`
def g(h, x):
    return h * f(x) / 2 + c * 2 * f(x) / h


# `设置h`
log_h = np.linspace(0, -20, 200)
h = 10 ** log_h

# `机器精度`
c = 6 * 10 ** -16

# `计算误差值`
error = np.array([])
for i in range(len(h)):
    temp = abs(derivative_f(0.5, h[i]) - real_df(0.5))
    error = np.append(error, np.log10(temp))

# `绘制图像`
plt.figure()
plt.plot(log_h, np.log10(g(h, 0.5)))
plt.scatter(log_h, error, s=5, color='red')
plt.show()

# `计算最佳的h,等价于g'(h)=0时的h的值`
h0=np.sqrt(4*c)
print(h0)
    \end{lstlisting}

    运行结果如下

    与示意图相同:
    \begin{figure}[h]
        \centering
        \includegraphics[height=0.6 \textheight]{Prob4c1}
        \caption{HW8\_4\_c 运行结果}\label{fig:figure4.3.1}
    \end{figure}

    \newpage
    重新绘图(Reproduce)(\(\varepsilon'' = 6 \times 10^{-16}\))
    \begin{figure}[h]
        \centering
        \includegraphics[height=0.6 \textheight]{Prob4c2}
        \caption{HW8\_4\_c 运行结果}\label{fig:figure4.3.2}
    \end{figure}
    \newpage
    最佳 \(h\) 值为:
    \begin{equation*}
        h = 4.898979485566356 \times 10^{-8}
    \end{equation*}

    \newpage
    \subsection{d}\label{subsec:4.4}
    如(b)问结论部分的叙述,可以给出两种误差的表达式:

    截断误差:
    \begin{equation*}
        e\_tru = h / 2 \abs{f''(x)}
    \end{equation*}

    舍入误差:
    \begin{equation*}
        e\_rou = (2c \abs{f(x)}) / h
    \end{equation*}
    其中 \(c\) 为机器精度

    进而写出总误差:
    \begin{equation*}
        e = e\_tru + e\_rou = h / 2 \abs{f''(x)} + (2c \abs{f(x)}) / h
    \end{equation*}

    可以发现总误差 \(e\) 存在一个最小值,而使 \(e\) 取到这一最小值的 \(h\) 值即为最佳 \(h\) 值,也即 \(h_{best}\) 。

    由均值不等式有:
    \begin{equation*}
        \frac{h}{2} \abs{f''(x)} + \frac{2c \abs{f(x)}}{h} \geq 2 \sqrt {\frac{c \abs{f(x)}}{\abs{f''(x)}}}
    \end{equation*}

    取等条件为:
    \begin{equation*}
        \frac{h}{2} \abs{f''(x)} = \frac{2c \abs{f(x)}}{h}
    \end{equation*}

    进而求得 \(h_{best}\):
    \begin{equation*}
        h_{best} = \sqrt {\frac{4c \abs{f(x)}}{|f''(x)|}}
    \end{equation*}

    使用 Python 代码实现如下
    \begin{lstlisting}[language=Python, breaklines = true,label={lst:lstlisting4.4}]
import matplotlib.pyplot as plt
import numpy as np


# `设置函数f(x)=sin(x)`
def f(x):
    return np.sin(x)


# `设置求解导数的公式`
def derivative_f(x, delta):
    return (f(x + delta) - f(x)) / delta


# `设置导数函数 f'(x)=cos(x)`
def real_df(x):
    return np.cos(x)


# `设置h`
log_h = np.linspace(0, -20, 200)
h = 10 ** log_h

# `机器精度`
c = 6 * 10 ** -16


# `计算截断误差`
def e_tru(h, x):
    return h * f(x) / 2


# `计算舍入误差`
def e_rou(h, x):
    return c * 2 * f(x) / h


# `计算总误差`
def g(h, x):
    return e_rou(h, x) + e_tru(h, x)


# `计算h的最佳取值`
def h_best(x):
    return np.sqrt(4 * c)


# `计算各种误差`
e_truncation = e_tru(h, 0.5)
e_roundoff = e_rou(h, 0.5)
e = g(h, 0.5)

# `绘制图像与输出结果`
plt.figure()
plt.plot(log_h, np.log10(e_truncation), label='truncation error')
plt.plot(log_h, np.log10(e_roundoff), label='round-off error')
plt.plot(log_h, np.log10(e), label='total error')
plt.xlabel('log10(h)')
plt.legend()
plt.show()

# `计算并输出h\_best`
print('`h的最佳取值为:`', '\n', h_best(0.5))
    \end{lstlisting}

    以(c)给出的机器精度:
    \begin{equation*}
        c = 6 \times 10^{-16}
    \end{equation*}
    为例。
    \begin{figure}[h]
        \centering
        \includegraphics[height=0.6 \textheight]{Prob4d}
        \caption{HW8\_4\_d 运行结果}\label{fig:figure4.4}
    \end{figure}

    \newpage
    \(h\) 的最佳取值为:
    \begin{equation*}
        4.898979485566356e-08
    \end{equation*}

\end{document}
